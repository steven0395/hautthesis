%%%%%%%%%%%%%%%%%%%%%%%%%%%%%%%%%%%%%%%%%%%%%%%%%%%%%%%%%%%%%%%%%%%%%%%
%% This is a thesis template for Henan University of Technology,
%% suitable for doctoral and academic master's degree students,
%% currently only supports Windows system.
%%%%%%%%%%%%%%%%%%%%%%%%%%%%%%%%%%%%%%%%%%%%%%%%%%%%%%%%%%%%%%%%%%%%%%%

%!TEX encoding = UTF-8 Unicode
%!TEX program = xelatex

% 若要让每一章首页均从奇数页开始,请修改openany为openright
\PassOptionsToClass{openany}{hautthesis}

% doctor, master, professional 对应 博士,学术硕士,专业硕士
\documentclass[doctor]{hautthesis}
% \usepackage{amsmath}
\usepackage{algorithmicx,algpseudocode}
\usepackage{float}
\usepackage{setspace}
\usepackage{pifont}
\usepackage[export]{adjustbox}


% 学校代码
\hautcode{10463}
% 学校名称
\hautname{河南工业大学}
\enhautname{Henan University of Technology}
% 中图分类号
\dclc{TH113}
% 国图分类号
\fclc{FTH113}
% 密级       
\secrettext{公开}           

% 标题
\title{储藏谷物品质***********及其评价方法研究}
\entitle{The Implementation of the Third Perpetual Motion Machine}
% 作者
\author{王某某}
\enauthor{Wang Moumou}
% 学号
\authorid{S18100001}      
% 学院
\college{粮油食品学院}
\encollege{College of Technology Food Science and and and Technology}
% 专业
\major{食品科学与工程} 
\enmajor{Computer Science and Technology}
% 学科门类
\subject{工学}
% 研究方向
\workon{粮食储藏理论与技术}

% 导师
\supervisor{刘某某\ 教授} % \后面的空格不可省略
\ensupervisor{Prof. Liu Moumou}
% 共同导师(如果有)
% \cosupervisor{张某某\ 教授}
% \encosupervisor{Prof. Zhang Moumou}
% 校外指导老师,专业硕士使用
% \outsupervisor{张某某\ 高级工程师}
% \enoutsupervisor{}

% 论文完成、答辩日期
\submitdate{二〇一八年五月}
\defensedate{二〇一八年六月十五日}
\endissertation{June 2, 2018}
% 答辩委员会主席
% \chair{待定}


\begin{document}
% 封面、中英文扉页、原创性声明、授权书
\maketitle

% 摘要
%中文摘要
\begin{abstract}
	中文摘要。
	\keywords{关键字1;关键字2;关键字3;关键字4;关键字5}
\end{abstract}

%英文摘要
\begin{enabstract}
	Abstract in English.
	\enkeywords{keyword1; keyword2; keyword3; keyword4; keyword5}
\end{enabstract}
% 目录
\tableofcontents

% 插图附表索引,根据需要开启
% \listoffigures
% \listoftables

% 正文章节
\mainmatter
\input{chapters/ch1}
\chapter{标题}

正文。
\input{chapters/ch3}
\input{chapters/summary}

% 致谢
\input{chapters/acknowledgements}

% 参考文献
\bibliography{references}

% 附录,根据需要插入
\hautappendix
\chapter{公式附录}

\chapter{图表附录}


% 发表的科研成果
\backmatter
\chapter{攻读博士学位期间发表的论文及科研成果}

\begin{enumerate}
    \item 第三类永动机
\end{enumerate}

\end{document}